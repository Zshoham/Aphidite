\documentclass{article}


\usepackage{../arxiv}

\usepackage[utf8x]{inputenc} % allow utf-8 input
\usepackage[T1]{fontenc}    % use 8-bit T1 fonts
\usepackage{enumerate}      % pretty enumeration
\usepackage[titletoc]{appendix}
\usepackage{microtype}      % microtypography
\usepackage{listings}
\usepackage{subfiles}

\title{Aphid Detection in Lettuce and Cannabis Leaves \\ Research Proposal}

\subfile{../authors}

\begin{document}
\maketitle

\section*{General Problem Description}
Aphids are among the most destructive insect pests, and monitoring aphid populations provides important data related to integrated pest management. Inefficient and labor-intensive is the manual identification of aphids. In order to reduce aphid crop damage, recent advances in computer vision algorithms - including object detection - can improve the process of identifying, classifying, and monitoring aphids.

\section*{Required Implementation}
For the purpose of identifying the aphids, we will use YOLO/RCNN with preprocessing (fracturing the images and choosing the most likely fragments based on the percentage of green pixels). Our work on improving the baseline results includes adding our own modifications (for example, adding more contrast to the colors in the image) to further enhance the results. We will then evaluate our modified algorithms based on the chosen evaluation metrics (described in the following section). We will simplify the problem by converting it to binary: 1 for aphids and 0 for no aphids.

\section*{General Description of Experiments}
\begin{enumerate}
  \item Dataset preprocessing:
  \begin{enumerate}
    \item <fill here when the data will be published> 
  \end{enumerate}
  \item Randomly split the dataset into train and test sets, 80-20 percent respectively.
  \item Tune hyperparameter using K-Fold cross validation.
  \item Train each algorithm on the training set, as paying attention to overfitting.
  \item Evaluate each model on the test set using appropriate metrics, like F1 and ROC AUC.
  \item Compare performance results.
\end{enumerate}

\section*{General Description of User Study}
Providing images of data with and without aphid, where some showing the error made by the model.
Users will be able to choose the images with the aphids in the following way:
\begin{enumerate}
  \item Every user will receive a set of images from one of the models without knowing which one it is.
  \item Their task is to lable the aphid in the images.
\end{enumerate}
As a result of the user study, we will be able to learn how much the model error is similar to a human error.
\end{document}
