\documentclass{article}

\usepackage{../arxiv}

\usepackage[utf8x]{inputenc} % allow utf-8 input
\usepackage[T1]{fontenc}    % use 8-bit T1 fonts
\usepackage{enumerate}      % pretty enumeration
\usepackage[titletoc]{appendix}
\usepackage{microtype}      % microtypography
\usepackage{listings}
\usepackage{subfiles}

\title{Aphid Detection in Lettuce and Cannabis Leaves \\ Research Proposal}

\subfile{../authors}

\begin{document}
\maketitle

\section*{General Problem Description}
Aphids are among the most destructive insect pests, and monitoring aphid populations 
provides essential data related to integrated pest management. 
Manual identification of aphids is inefficient and labour-intensive. 
Recent advances in computer vision algorithms , including object detection, 
can improve the identification, classification, 
and monitoring of aphids to reduce crop damage.

\section*{Required Implementation}
To identify aphids, we will use YOLO/RCNN with preprocessing
(fracturing the images and choosing the most likely fragments based on the percentage of green pixels).
Our work on improving the baseline results includes adding our own modifications
(for example, adding more contrast to the colors in the image or filling image borders with some pixel values)
to enhance the results further. We will then evaluate our modified algorithms
based on the chosen evaluation metrics (described in the following section).
We will simplify the problem by converting it to binary: 1 for aphids and 0 for no aphids.


\section*{General Description of Experiments}
\begin{enumerate}
  \item Dataset preprocessing:
  \begin{enumerate}
    \item Resize image into 256*256 pixels.
    \item Delete all the images with less than some percent of green pixels. The percentage will be a hyperparameter.
    \item In our model, apply augmentation.
    \item Normalize the data.
  \end{enumerate}
  \item Randomly split the dataset into train and test sets, 80-20 percent respectively.
  \item Tune hyperparameter using K-Fold cross-validation.
  \item Train each algorithm on the training set.
  \item Evaluate each model on the test set using appropriate metrics, like F1 and ROC AUC.
  \item Compare performance results.
\end{enumerate}

\begin{samepage}
  \section*{General Description of User Study}
  Providing images of data with and without aphid, where some showing the error made by the model.
  Users will rank model's aphids prediction in the following way:
  \begin{enumerate}
    \item Every user will receive a set, contains pairs of images. Each pair is the same picture where the difference is the labeling - each one of the pictures is labeled by different models. The user will not know which model tagged which image.
    \item The task is to choose which labeling is more successful.
  \end{enumerate}
  As a result of the user study, we will be able to learn how much the model error is similar to a human error.
\end{samepage}
\end{document}
